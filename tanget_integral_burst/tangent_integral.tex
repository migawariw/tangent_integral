% \documentclass[a4j,7pt,uplatex,dvipdfmx,twocolumn,fleqn]{jsarticle}
\documentclass[a4j,uplatex,dvipdfmx,10pt]{jsarticle}
\usepackage{mypackages}
\usepackage{aleftright}
\csname endofdump\endcsname
\usepackage[italicdiff,burboxthm,geqq]{myset}
% \def\drac#1#2{\mathchoice{\displaystyle\fracdefcopied{\hspace{.15em}#1\hspace{.15em}}{\hspace{.15em}#2\hspace{.15em}}}{\displaystyle\fracdefcopied{\hspace{.15em}#1\hspace{.15em}}{\hspace{.15em}#2\hspace{.15em}}}{\scriptstyle\fracdefcopied{#1}{#2}}{\scriptscriptstyle\fracdefcopied{#1}{#2}}}
\usepackage{bigints}
% \def\cdrac#1#2{\begin{array}{cc}\drac{#1}{#2}\end{array}}
\mathtoolsset{showonlyrefs=true}
\let\abovedisplayskipdefcopied\abovedisplayskip
\let\belowdisplayskipdefcopied\belowdisplayskip
\def\autoleft{\l}
\def\autoright{\r}
\AtBeginDocument{
% \setlength{\abovedisplayskip}{1\abovedisplayskipdefcopied}%
% \setlength{\belowdisplayskip}{0\belowdisplayskipdefcopied}%
}
% \geometry{top=20mm,bottom=20mm,left=10mm,right=10mm}
\DeclareDocumentEnvironment{lemm}{O{}}{\begin{mygraybox}{}\begin{lem}[#1]}{\end{lem}\end{mygraybox}}%
\let\orfrac\drac
\title{\(\tan ^{\drac{m}{n}}\theta  \)の不定積分}
\author{みがわり\\ migawariw@gmail.com}
\date{\today}
\begin{document}
\maketitle
\tableofcontents
\section{はじめに}
\label{sec:はじめに}
今回はYouTubeのヨビノリたくみさんのチャンネル\href{https://www.google.com/url?sa=t&rct=j&q=&esrc=s&source=video&cd=&cad=rja&uact=8&ved=2ahUKEwiar7fIzpb6AhUDw4sBHU28DxMQFnoECAkQAg&url=https://www.youtube.com/channel/UCqmWJJolqAgjIdLqK3zD1QQ&usg=AOvVaw3clbmmW3q8O9-Fgp4nIJlw}{予備校のノリで学ぶ「大学の数学・物理」}の「今週の積分」の最後に取り上げられていた以下の積分の問題を一般化してみました。
\begin{equation}
\int_{0}^{\drac{\pi }{4}}\sqrt{\tan x}\dd{x}
\label{eqn:2022-09-15_19-44-54}
\end{equation}
具体的には、任意の有理数\(r \)に対して\(\int_{}^{} \tan ^{r}x \dd{x} \)を初等関数の範囲で求める方法を導出しました。特に絶対値が\(1 \)未満の有理数\(r \)については\(\tan ^r x \)の不定積分の具体的な形を求めています。内容的には高3のときに考えたものです。そのため\(\arctan  \)\footnote{一応説明は書きましたがそれでもわからない方は調べていただければ結構出てくると思います。\label{fot:2022-09-15_16-33-51}}を最後に使うこと以外は高校範囲内の内容になっています。高校生の方もご安心下さい。

方針としては部分分数分解した後、各項を積分するというシンプルなものです。途中にド・モアブルの公式や恒等式等の知識が必要となることを述べておきます。
% 実はこの記事、ほとんどが三角関数の式変形についての内容になっています。

不定積分の証明は\secref{sec:不定積分の証明}の中だけで完結しています。不定積分のみわかれば十分という方はそこだけお読みください。\\

\secref{sec:定積分}では指数の絶対値が\(1 \)を超えない場合の\(0 \)から\(\drac{\pi }{2} \)および\(0 \)から\(\drac{\pi }{4} \)の範囲の定積分の値を求めています。\secref{sec:その他のケース}では指数の絶対値が\(1 \)を超える場合について考察しています。
% その際\secref{sec:補足;三角関数の総和公式について}三角関数の総和公式についての内容が必要になります。内容の独立性が強いことから分離しました。

\secref{sec:元ネタ}以下後半では他の問題への応用例(ヨビノリさんの問題含む)を掲載しました。そちらも是非ご覧ください。
\section{不定積分の証明(\(\l|m\r|<n \)のとき)}
\label{sec:不定積分の証明}
以下、\(\theta  \in \mathbb{R}\)として\(\theta  \)が動く範囲を、\(\tan^{\drac{m}{n}} \theta  \)が発散する値をまたぐことがない範囲に限定して考える\footnote{例えば\(m>0 \)として\(\drac{3}{4}\pi \leq \theta \leq \pi  \)の範囲であれば考えるが\(0\leq \theta \leq \pi  \)のような積分範囲は考えない(この場合は\(\theta =\drac{\pi }{2} \)をまたがない)ということ。\label{fot:2022-09-15_17-45-06}}。この記事全体において\(m \in \mathbb{Z},n \in \mathbb{N} \)とする。\(x \)の関数\(\arctan x \)を\(y=\tan x \)の\(-\drac{\pi }{2}<x<\drac{\pi }{2} \)の範囲における逆関数とする。
% \(k,p,q \in \mathbb{Z} \)として一般に数列\(\l\{a_k\r\} \)に対して\(\prod_{k=p}^{q} a_k \)を\(a_p \cdot a_{p+1}\cdots a_{q-1}\cdot a_q \)とする。
\begin{thm}[不定積分]
\label{thm:2022-09-12_23-29-44}
\(m \in \mathbb{Z},n \in \mathbb{N} \)として、\(\l|m\r|<n \)のもとで以下の等式が成り立つ。
\begin{equation}\begin{split}
 \int_{}^{} \tan ^{\drac{m}{n}}\theta \dd{\theta }=
 \begin{multlined}[t]
 \sum_{k=1}^{n} \l[-\drac{1}{2}\cos \l(n+m\r)\theta _{n,k} \log \l(\tan ^{\drac{2}{n}}\theta -2 \cos \theta _{n,k} \tan ^{\drac{1}{n}}\theta +1\r)\r. \\
 \l. +\sin \l(n+m\r)\theta _{n,k}\arctan \l(\drac{\tan ^{\drac{1}{n}}\theta -\cos \theta _{n,k}}{\sin \theta _{n,k}}\r)
\r]+C \\
 \l(\theta _{n,k}\coloneqq \drac{2k-1}{2n}\pi \hspace{.17em}とした。Cは積分定数。\r)
 \end{multlined}
\end{split}\label{eqn:2022-09-12_23-31-22}
\end{equation}
\end{thm}

\begin{prf}[]
\(x\coloneqq \tan ^{\drac{1}{n}}\theta  \)とおく。
\begin{align}
\dv{x}{\theta }= {} & \drac{1}{n}\tan ^{\drac{1}{n}-1}\theta \l(1+\tan ^2\theta \r)=\drac{1}{n}x^{1-n}\l(1+x^{2n}\r)
\label{eqn:2022-09-12_23-35-03}
\end{align}
より
\begin{equation}
\dd{\theta }=\drac{nx^{n-1}}{1+x^{2n}}\dd{x}
\label{eqn:2022-09-12_23-37-27}
\end{equation}
よって
\begin{equation}
\int_{}^{} \tan ^{\drac{m}{n}}\theta \dd{\theta }=\int_{}^{} x^m\cdot \drac{nx^{n-1}}{1+x^{2n}}\dd{x}=n \int_{}^{} \drac{x^{n+m-1}}{1+x^{2n}}\dd{x}
\label{eqn:2022-09-12_23-38-28}
\end{equation}




\begin{lemm}[部分分数分解の公式]
\label{lem:2022-09-12_23-41-12}
\(k \in \mathbb{Z} ,e\l(\theta \r)\coloneqq \cos \theta +i \sin \theta \)とする。\\
\(\theta _{n,k}\coloneqq \drac{2k-1}{2n}\pi ,z_{n,k}\coloneqq e\l(\theta _{n,k}\r) \)とおく。\(\l|m\r|<n \)のもとで\(z \in \mathbb{C}\)に対して以下の等式が成り立つ。
\begin{equation}
\drac{z^{n+m-1}}{1+z^{2n}}=-\drac{1}{2n}\sum_{k=1}^{2n} \drac{z_{n,k}^{n+m}}{z-z_{n,k}}
\label{eqn:2022-09-12_23-44-16}
\end{equation}

\end{lemm}

\begin{prf}[\lemref{lem:2022-09-12_23-41-12}]
方程式\(1+z^{2n}=0 \)の解を考える。\(r>0,0\leq \theta < 2\pi  \)として\(z=r\l( \cos \theta + i \sin \theta \r) \)とおく。ド・モアブルの公式より
\begin{equation}
1+z^{2n}=0 \qq{すなわち} r^{2n}\l(\cos 2n\theta +i \sin 2 n \theta \r)=\cos \pi +i \sin \pi
\label{eqn:2022-09-13_01-36-13}
\end{equation}
である。絶対値と偏角を比較して
\begin{align}
r=1\qc 2n\theta =\pi +2k\pi
\label{eqn:2022-09-13_01-37-42}
\end{align}
\(0\leq \theta < 2\pi  \)より
\begin{equation}
\theta =\drac{2k-1}{2n}\pi=\theta _{n,k} \quad \l(k=1,\ldots ,2n\r)
\label{eqn:2022-09-13_01-39-50}
\end{equation}
よって解は\(z=z_{n,k} \)(\(k=1,\ldots ,2n \))である。

以下のように部分分数分解されると仮定する。
\begin{equation}
\drac{z^{n+m-1}}{1+z^{2n}}=\sum_{k=1}^{2n} \drac{a_{n,m,k}}{z-z_{n,l}}
\label{eqn:2022-09-12_23-48-56}
\end{equation}
すなわち、一般に数列\(\l\{a_k\r\} \)に対して\(\prod_{\stackrel{1\leq l\leq N}{l \neq k}}^{}a_l\coloneqq a_1\cdot a_2\cdots a_{k-1}\cdot a_{k+1}\cdots a_{N} \)と定めて
\begin{equation}
z^{n+m-1}=\sum_{k=1}^{2n} a_{n,m,k}\prod_{\stackrel{1\leq l\leq 2n}{l \neq k}}^{} \l(z-z_{n,l}\r)
\label{eqn:2022-09-12_23-51-25}
\end{equation}
が恒等式である仮定する。そうなるには\(p =1,\ldots ,2n \)として\(z=z_{n,p} \)を代入した
\begin{align}
z_{n,p}^{n+m-1}= {} & \sum_{k=1}^{2n} a_{n,m,k}\prod_{\stackrel{1\leq l\leq 2n}{l \neq k}}^{} \underbrace{\l(z_{n,p}-z_{n,l}\r)}_{k=p\text{のとき以外どれかが0になる}} = a_{n,m,p}\prod_{\stackrel{1\leq l\leq 2n}{l \neq p}}^{}\l(z_{n,p}-z_{n,l}\r)
\label{eqn:2022-09-12_23-55-12}
\end{align}
が成立する必要がある。ここで\(z\neq z_{n,p} \)において以下の等式が常に成り立つことに注意する。
\begin{align}
\prod_{\stackrel{1\leq l\leq 2n}{l \neq p}}^{}\l(z-z_{n,l}\r)= 
\drac{z^{2n}+1}{z-z_{n,p}}=\drac{z^{2n}-z_{n,p}^{2n}}{z-z_{n,p}}={}& \drac{\l(z/z_{n,p}\r)^{2n}-1}{\l(z/z_{n,p}\r)-1}\cdot z_{n,p}^{2n-1} 
= \sum_{l'=0}^{2n-1} \l(\drac{z}{z_{n,p}}\r)^{l'}\cdot \drac{-1}{z_{n,p}}\\
\therefore\prod_{\stackrel{1\leq l\leq 2n}{l \neq p}}^{}\l(z-z_{n,l}\r)                                                   ={}         & \sum_{l'=0}^{2n-1} \l(\drac{z}{z_{n,p}}\r)^{l'}\cdot \drac{-1}{z_{n,p}}
\label{eqn:2022-09-12_23-59-55}
\end{align}
最後の変形において等比数列の和の公式と\(z_{n,p}^{2n}=-1 \)であることを用いた。\(z=z_{n,p} \)以外の任意の複素数について成立するということは\(2n \)個以上の解をもつということだから\eqref{eqn:2022-09-12_23-59-55}は恒等式である。したがって\(z=z_{n,p} \)を代入した
\begin{equation}
\prod_{\stackrel{1\leq l\leq 2n}{l \neq p}}^{}\l(z_{n,p}-z_{n,l}\r)=\sum_{l'=0}^{2n-1} \l(\drac{z_{n,p}}{z_{n,p}}\r)^{l'}\cdot \drac{-1}{z_{n,p}}=-\drac{2n}{z_{n,p}}
\label{eqn:2022-09-13_00-05-02}
\end{equation}
が成り立つ。したがって\eqref{eqn:2022-09-12_23-55-12}は
\begin{equation}
z_{n,p}^{n+m-1}=a_{n,m,p}\cdot \l(-\drac{2n}{z_{n,p}}\r)\qq{すなわち} a_{n,m,p}=-\drac{1}{2n}z_{n,p}^{n+m}
\label{eqn:2022-09-13_00-08-13}
\end{equation}
と言い換えられて、\(p=1,\ldots ,2n \)においてこれが成り立つことが必要条件であるとわかる。逆に\(p \)を\(k \)に変えて\(a_{n,m,k}=-\drac{1}{2n}z_{n,k}^{n+m} \)を代入した
\begin{equation}
z^{n+m-1}=\sum_{k=1}^{2n} \autoleft[\begin{array}{cc}
-\drac{1}{2n}z_{n,k}^{n+m}\prod_{\stackrel{1\leq l\leq 2n}{l \neq k}}^{}\l(z-z_{n,l}\r)
\end{array}\autoright]
\label{eqn:2022-09-13_00-13-15}
\end{equation}
は\(z=z_{n,p} \)を各々代入することによって\(p=1,\ldots ,2n \)の\(2n \)個の\(z_{n,p} \)が確かに解になることがわかる。すると、\(\l|m\r|<n \)により\eqref{eqn:2022-09-13_00-13-15}は両辺いずれも\(2n-1 \)次以下の式であるはずなのに\(p=1,\ldots ,2n \)の\(2n \)個の\(z=z_{n,p} \)を解に持っていることになる。したがって確かに恒等式であるとわかる。すると\eqref{eqn:2022-09-13_00-13-15}の両辺\(1+z^{2n} \)で割った
\begin{equation}
\drac{z^{n+m-1}}{1+z^{2n}}=-\drac{1}{2n}\sum_{k=1}^{2n} \drac{z_{n,k}^{n+m}}{z-z_{n,k}}
\label{eqn:2022-09-13_00-14-55}
\end{equation}
が得られる。
\end{prf}
引き続き\(z \)のまま積分の中身を変形する。
\begin{equation}\begin{split}
\drac{z^{n+m-1}}{1+z^{2n}} ={} & -\drac{1}{2n}\sum_{k=1}^{2n} \drac{z_{n,k}^{n+m}}{z-z_{n,k}} = -\drac{1}{2n}\sum_{k=1}^{n} \l(\drac{z_{n,k}^{n+m}}{z-z_{n,k}}+\drac{z_{n,2n+1-k}^{n+m}}{z-z_{2n+1-k}}\r)
\end{split}\label{eqn:2022-09-13_00-17-14}
\end{equation}
ここで
\begin{equation}
\theta _{n,2n+1-k}=\drac{2\l(2n+1-k\r)-1}{2n}\pi =2\pi -\drac{2k-1}{2n}=2\pi -\theta _{n,k}
\label{eqn:2022-09-13_00-29-39}
\end{equation}
より\(z_{n,2n+1-k}=z_{n,k}^{-1} \)であり
\begin{align}[left=\empheqlbrace\ ]
{} & z_{n,k}+z_{n,2n+1-k}=2 \cos \theta _{n,k}
\label{eqn:2022-09-13_00-32-23} \\
{} & z_{n,k}z_{n,2n+1-k}=1
\label{eqn:2022-09-13_00-31-21}
\end{align}
である。また
\begin{equation}\begin{split}
{}   & z_{n,k}^{n+m}z_{n,2n+1-k}+z_{n,2n+1-k}^{n+m}z_{n,k}=z^{n+m-1}_{n,k}+z_{n,k}^{-\l(n+m-1\r)} \\
= {} & e\l(\l(n+m-1\r)\theta _{n,k}\r)+e\l(-\l(n+m-1\r)\theta _{n,k}\r)=2\cos \l(n+m-1\r)\theta _{n,k}
\end{split}\label{eqn:2022-09-13_00-34-46}
\end{equation}
より分母を通分して
\begin{equation}\begin{split}
\drac{z^{n+m-1}}{1+z^{2n}}= {} & -\drac{1}{2n}\sum_{k=1}^{n} \drac{2z \cos \l(n+m\r)\theta _{n,k}-2 \cos \l(n+m-1\r)\theta _{n,k}}{z^{2}-2 \cos \theta _{n,k}z+1} \\
= {}                           & 
-\drac{1}{2n}\sum_{k=1}^{n}\autoleft[ \begin{array}{cc}\drac{\l(2z-2 \cos \theta _{n,k}\r)\cos \l(n+m\r)\theta _{n,k}}{z^{2}-2 \cos \theta _{n,k}z+1}+2\  \drac{\cos \theta _{n,k}\cos \l(n+m\r)\theta _{n,k}-\cos \l(n+m-1\r)\theta _{n,k}}{\l\{\l(\orfrac{z- \cos \theta _{n,k}}{\sin \theta _{n,k}}\r)^{2}+1\r\}\sin ^2\theta _{n,k}}
\end{array}\autoright]
\end{split}\label{eqn:2022-09-13_00-37-23}
\end{equation}
ここで加法定理より
% \begin{equation}\begin{split}
% {}   & \cos \theta _{n,k}\cos \l(n+m\r)\theta _{n,k}-\cos \l(n+m-1\r)\theta _{n,k} \\
% = {} & \cos \theta _{n,k}\l(\cos \l(n+m-1\r)\theta _{n,k }\cos \theta _{n,k}-\sin \l(n+m-1\r)\theta _{n,k}\sin \theta _{n,k}\r)-\cos \l(n+m-1\r)\theta _{n,k} \\
% = {} & \l(\cos ^2\theta _{n,k}-1\r)\cos \l(n+m-1\r)\theta _{n,k}-\cos \theta _{n,k}\sin \l(n+m-1\r)\theta _{n,k}\sin \theta _{n,k} \\
% = {} & - \sin \theta _{n,k}\l\{\sin \theta _{n,k }\cos \l(n+m-1\r)\theta _{n,k}+\sin \l(n+m-1\r)\theta _{n,k}\cos \theta _{n,k}\r\} \\
% = {} & - \sin \theta _{n,k}\sin \l(n+m \r)\theta _{n,k}
% \end{split}\label{eqn:2022-09-13_00-54-13}
% \end{equation}
\begin{align}
\cos \l(n+m-1\r)\theta _{n,k}=\cos \theta _{n,k}\cos \l(n+m\r)\theta _{n,k}+\sin \theta _{n,k}\sin \l(n+m\r)\theta _{n,k}
\label{eqn:2022-09-15_16-02-55}
\end{align}
であるから第2項の分子は
\begin{equation}
\cos \theta _{n,k}\cos \l(n+m\r)\theta _{n,k}-\cos \l(n+m-1\r)\theta _{n,k}=-\sin \theta _{n,k}\sin \l(n+m\r)\theta _{n,k}
\label{eqn:2022-09-15_16-04-01}
\end{equation}
と変形できる。\(z \)を\(x \)に戻して、見やすいようにいくつか積の順番を入れ替えると\eqref{eqn:2022-09-12_23-38-28}と合わせて
\begin{equation}\begin{split}
{} & \int_{}^{} \tan ^{\drac{m}{n} }\theta\dd{\theta }\xlongequal{x= \tan ^{\drac{1}{n}}\theta }
                                          -\drac{n}{2n}\int_{}^{} \sum_{k=1}^{n}\begin{multlined}[t] \l[\cos \l(n+m\r)\theta _{n,k}\drac{2x-2 \cos \theta _{n,k}}{x^2-2x \cos \theta _{n,k}+1}\r. \\
                                          \l.\begin{array}{cc}
																					 +2\drac{- \sin \theta _{n,k} \sin \l(n+m\r)\theta _{n,k}}{\sin ^2\theta _{n,k}}\drac{1}{\l(\orfrac{z- \cos \theta _{n,k}}{\sin \theta _{n,k}}\r)^2+1}
																					\end{array}\r]\dd{x}
                                          \end{multlined}
\end{split}\label{eqn:2022-09-13_01-07-44}
\end{equation}
と変形できる。積分を\(\sum \)の中に入れると
\begin{equation}\begin{split}
(上式)= {}
    & 
      -\drac{1}{2}\sum_{k=1}^{n} \begin{multlined}[t]\l[\cos \l(n+m\r)\theta _{n,k}\int_{}^{} \drac{\l(x^2-2x \cos \theta _{n,k}+1\r)'}{x^2-2x \cos \theta _{n,k}+1}\dd{x}\r. \\
      \l. -2 \sin \l(n+m\r)\theta _{n,k}\bigintss\begin{array}{cc}
			\cfrac{1}{1+\l(\drac{x- \cos \theta _{n,k} }{\sin \theta _{n,k}}\r)^2}
			\end{array}\drac{\dd{x}}{\sin \theta _{n,k}}\r]
      \end{multlined} \\
={} & 
      -\drac{1}{2}\sum_{k=1}^{n} \begin{multlined}[t]\l[\cos \l(n+m\r)\theta _{n,k}\log\l({x^2-2x \cos \theta _{n,k}+1}\r)\r. \\
      \l. -2 \sin \l(n+m\r)\theta _{n,k}\arctan \l(\drac{x- \cos \theta _{n,k}}{\sin \theta _{n,k}}\r)\r]+C
      \end{multlined} \\
={} & 
      \sum_{k=1}^{n} \begin{multlined}[t]\l[-\drac{1}{2}\cos \l(n+m\r)\theta _{n,k}\log\l({\tan ^{\drac{2}{n}}\theta -2\cos \theta _{n,k}\tan ^{\drac{1}{n}}\theta  +1}\r)\r. \\
      \l. + \sin \l(n+m\r)\theta _{n,k}\arctan \l(\drac{\tan ^{\drac{1}{n}}\theta - \cos \theta _{n,k}}{\sin \theta _{n,k}}\r)\r]+C
      \end{multlined} \\
\end{split}\label{eqn:2022-09-13_01-20-09}
\end{equation}
が導かれた。
\end{prf}
以降、この不定積分から積分定数\(C \)を除いたものを\(f\l(\theta \r) \)とおく。

\section{定積分(\(\l|m\r|<n \)のとき)}
\label{sec:定積分}

\subsection{指数の分子が正のとき}
\label{ssc:指数の分子が正のとき}


ここでは簡単のため\(m>0 \)の範囲に限って考える。先にこの先の証明において必要になる定理を示しておく。\(e\l(\theta \r)\coloneqq \cos \theta +i \sin \theta  \)とする。
\begin{thm}[三角関数の総和公式]
\label{thm:2022-09-13_21-34-14}
\(\theta ,\varphi  \in \mathbb{R},n \in \mathbb{N}\)とする。以下の等式が成り立つ。
\begin{align}
\sum_{k=1}^{n} \cos \l(k\theta +\varphi \r)= {} & \drac{\sin \orfrac{n}{2}\theta \cos \l(\orfrac{n+1}{2}\theta +\varphi \r)}{\sin \orfrac{\theta }{2}}
& 
\sum_{k=1}^{n}
\sin \l(k\theta +\varphi \r)= {}                 \drac{\sin \orfrac{n}{2}\theta \sin \l(\orfrac{n+1}{2}\theta +\varphi \r)}{\sin \orfrac{\theta }{2}}
\label{eqn:2022-09-13_21-34-18}
\end{align}

\end{thm}
\begin{prf}[]
\begin{equation}\begin{split}
{}   & \sum_{k=1}^{n} \cos \l(k\theta +\varphi \r)+i \sum_{k=1}^{n} \sin  \l(k\theta +\varphi \r) 
= {}  \sum_{k=1}^{n} \l[\cos \l(k\theta +\varphi \r)+i  \sin  \l(k\theta +\varphi \r)\r]=\sum_{k=1}^{n}  e\l(k\theta +\varphi \r) \\
={}  & \sum_{k=0}^{n-1}  e\l(\varphi +\theta \r)e\l(\theta \r)^k \quad \text{(ド・モアブルの公式より)}
\\
= {} & e\l(\varphi +\theta \r)\drac{1-e\l(\theta \r)^n}{1-e\l(\theta \r)}\quad \text{(等比数列の和の公式より)} \\
= {} & e\l(\varphi +\theta \r)\drac{1-e\l(n\theta \r)}{1-e\l(\theta \r)}
\end{split}\label{eqn:2022-09-13_21-38-27}
\end{equation}
である。ここで一般に
\begin{equation}\begin{split}
1-e\l(\theta \r)= {} & 1-\l(\cos \theta + i \sin \theta \r)=\l(1-\cos \theta \r)-i \sin \theta 
= {}                 2 \sin ^2\drac{\theta }{2}-2i \sin \drac{\theta }{2}\cos \drac{\theta }{2} \\
= {}                 & -2i \sin \drac{\theta }{2}\l(\cos \drac{\theta }{2}+i \sin \drac{\theta}{2} \r) 
= {}                  -2i \sin \drac{\theta }{2}e\l(\drac{\theta }{2}\r)
\end{split}\label{eqn:2022-09-13_21-46-47}
\end{equation}
であることを用いると\eqref{eqn:2022-09-13_21-38-27}の式は
\begin{equation}\begin{split}
{} & e\l(\varphi+\theta  \r)\drac{-2i \sin \orfrac{n}{2}\theta e\l(\orfrac{n}{2}\theta \r)}{-2i \sin \orfrac{\theta }{2}e\l(\orfrac{\theta }{2}\r)}=e\l(\varphi +\theta \r)\drac{e\l(\orfrac{n}{2}\theta \r)}{e\l(\orfrac{\theta }{2}\r)}\cdot \drac{ \sin \orfrac{n}{2}\theta }{ \sin \orfrac{\theta }{2}}=\drac{\sin \orfrac{n}{2}\theta }{ \sin \orfrac{\theta }{2}}e\l( \drac{n+1}{2} \theta+ \varphi\r)
\end{split}\label{eqn:2022-09-13_21-50-20}
\end{equation}
と変形できる。はじめの式と実部と虚部を比較して
\begin{align}
\sum_{k=1}^{n} \cos \l(k\theta +\varphi \r)= {} & \drac{\sin \orfrac{n}{2}\theta \cos \l(\orfrac{n+1}{2}\theta +\varphi \r)}{\sin \orfrac{\theta }{2}}
& 
\sum_{k=1}^{n}
\sin \l(k\theta +\varphi \r)= {}                & \drac{\sin \orfrac{n}{2}\theta \sin \l(\orfrac{n+1}{2}\theta +\varphi \r)}{\sin \orfrac{\theta }{2}}
\label{eqn:2022-09-13_22-19-49}
\end{align}
を得る。

\end{prf}
\begin{thm}[\(0 \)から\(\orfrac{\pi }{2} \)までの定積分]
\label{thm:2022-09-13_15-53-45}
\(m \in \mathbb{Z},n \in \mathbb{N} \)として、\(\l|m\r|<n \)かつ\(m>0 \)のもとで以下の等式が成り立つ。
\begin{align}
\lim_{\theta _1 \to \drac{\pi }{2}-0} \int_{0}^{\theta _1} \tan ^{\drac{m}{n}}\theta \dd{\theta }=\drac{\pi }{2 \cos \l(\orfrac{m}{n}\cdot \orfrac{\pi }{2}\r)} {} &
\label{eqn:2022-09-13_15-53-57}
\end{align}


\end{thm}
\begin{prf}[]
\begin{equation}
f\l(0\r)=\begin{multlined}[t]
         \sum_{k=1}^{n} \l[-\drac{1}{2}\cos \l(n+m\r)\theta _{n,k}\log\l({\tan ^{\drac{2}{n}}0 -2\tan ^{\drac{1}{n}}0  \cos \theta _{n,k}+1}\r)\r. \\
         \l. + \sin \l(n+m\r)\theta _{n,k}\arctan \l(\drac{\tan ^{\drac{1}{n}}0 - \cos \theta _{n,k}}{\sin \theta _{n,k}}\r)\r]
         \end{multlined}
\label{eqn:2022-09-13_18-34-16}
\end{equation}
第\(1 \)項は\(\log  \)の中身が\(1 \)なので\(0 \)。ここで
\begin{equation}
-\drac{\cos \theta _{n,k}}{\sin\theta _{n,k}}=\drac{\sin \l(\theta _{n,k}-\orfrac{\pi }{2}\r)}{\cos \l(\theta _{n,k}-\orfrac{\pi }{2}\r)}=\tan \l(\theta _{n,k}-\drac{\pi }{2}\r)
\label{eqn:2022-09-13_18-36-31}
\end{equation}
であり\(1\leq k\leq n \)において\(0<\theta _{n,k}=\drac{2k-1}{2n}\pi <\pi  \)より\(-\drac{\pi }{2}<\theta _{n,k}-\drac{\pi }{2}<\drac{\pi }{2} \)であることに注意して
\begin{equation}
f\l(0\r)=\sum_{k=1}^{n} \sin \l(\l(n+m\r)\theta _{n,k}\r)\cdot \l(\theta _{n,k}-\drac{\pi }{2}\r)
\label{eqn:2022-09-13_18-40-44}
\end{equation}
である。
% 以下の等式が成り立つ。
% \begin{equation}
% \sum_{k=1}^{n}\cos \l(n+m\r) \theta _{n,k} =0
% \label{eqn:2022-09-15_16-46-25}
% \end{equation}
% \lemref{lem:2022-09-15_16-45-59}は\secref{sec:補足;三角関数の総和公式について}で証明する。
\(\theta_0 =\drac{n+m}{2n}\pi  \)として\thmref{thm:2022-09-13_21-34-14}において\(\theta ,\varphi  \)をそれぞれ\(2\theta_0 ,-\theta_0  \)とすると
\begin{equation}
\sum_{k=1}^{n} \cos \l(n+m\r)\theta _{n,k}=\sum_{k=1}^{n} \cos \l(2k-1\r)\theta_0 =\drac{\sin n\theta_0 }{\sin \theta_0 }\cos n\theta_0 =\drac{\sin 2n\theta_0 }{\sin \theta_0 }=\drac{\sin \l(n+m\r)\pi }{\sin \theta_0 }=0
\label{eqn:2022-09-13_22-25-53}
\end{equation}
であるからこれを用いると
\begin{align}
& \lim_{\theta _1 \to \drac{\pi }{2}-0}f\l(\theta _{1}\r) \\
=  {}& \lim_{\theta _1 \to \drac{\pi }{2}-0}\begin{multlined}[t]
   \sum_{k=1}^{n} \l[-\drac{1}{2}\cos \l(n+m\r)\theta _{n,k}\log\l({\tan ^{\drac{2}{n}}\theta _1 -2\tan ^{\drac{1}{n}}\theta _1  \cos \theta _{n,k}+1}\r)\r. \\
   \l. + \sin \l(n+m\r)\theta _{n,k}\arctan \l(\drac{\tan ^{\drac{1}{n}}\theta _1 - \cos \theta _{n,k}}{\sin \theta _{n,k}}\r)\r]
   \end{multlined}
\label{eqn:2022-09-13_18-46-05} \\
={}    & \lim_{\theta _1 \to \drac{\pi }{2}-0}\begin{multlined}[t]
   \autoleft(\sum_{k=1}^{n} \autoleft[-\drac{1}{2}\cos \l(n+m\r)\theta _{n,k}\underbrace{\log\l(1-2 \tan ^{-\drac{1}{n}}\theta _1+\tan ^{-\drac{2}{n}}\theta _1\r)}_{\to 0}\autoright.\autoright.                                              \\
   \autoleft.\autoleft. + \sin \l(n+m\r)\theta _{n,k}\underbrace{\arctan \l(\drac{\tan ^{\drac{1}{n}}\theta _1 - \cos \theta _{n,k}}{\sin \theta _{n,k}}\r)}_{\sin \theta _{n,k}>0かつ\tan \theta _{1}\to +\infty より\to \drac{\pi }{2}}\autoright] +\log \l(\tan^{\drac{2}{n}}\theta _1\r)\underbrace{\sum_{k=1}^{n} -\drac{1}{2}\cos \l(n+m\r) \theta _{n,k}}_{={}0}\autoright)
   \end{multlined} \\
= {}   & \sum_{k=1}^{n} \drac{\pi }{2}\sin \l(n+m\r)\theta _{n,k}
\label{eqn:2022-09-13_18-42-54}
\end{align}
である\footnote{\(\sum_{k=1}^{n} -\drac{1}{2}\cos \l(n+m\r) \theta _{n,k} \)は「\(\theta _1 \to \drac{\pi }{2} \)で\(0 \)に漸近する」のではなく「\(0 \)そのものになる」ので不定形ではないことに注意。\label{fot:2022-09-15_16-21-44}}。したがって
\begin{equation}\begin{split}
\lim_{\theta _1 \to \drac{\pi }{2}-0}\int_{0}^{\theta _1}\tan ^{\drac{m}{n}}\theta  \dd{\theta }= {} & \sum_{k=1}^{n} \drac{\pi }{2}\sin \l(n+m\r)\theta _{n,k}-\sum_{k=1}^{n} \l(\theta _{n,k}-\drac{\pi }{2}\r)\sin \l(n+m\r)\theta _{n,k} \\
= {}& \sum_{k=1}^{n} \l(\pi -\theta _{n,k}\r)\sin \l(n+m\r)\theta _{n,k} \\
= {}& \sum_{k=1}^{n} \l(\pi -\drac{2k-1}{2n}\pi \r)\sin \l(\drac{n+m}{n}\drac{2k-1}{2}\pi \r)
\end{split}\label{eqn:2022-09-13_18-47-16}
\end{equation}
が得られる。ここで\(l=n-k+1 \)とおくと
\begin{equation}\begin{split}
\l(\text{上式}\r)= {} & \sum_{l=1}^{n} \l(\pi -\drac{2\l(n-l+1\r)-1}{2n}\pi \r)\sin \l(\drac{n+m}{n}\drac{2\l(n-l+1\r)-1}{2}\pi \r) \\
= {}                & \drac{\l(-1\r)^{n+m+1}\pi }{2n}\sum_{l=1}^{n} \l(2l-1\r)\sin \l(2l-1\r)\theta \quad \l(\theta =\drac{n+m}{2n}\pi \hspace{.17em}\text{とした。}\r)
\end{split}\label{eqn:2022-09-13_18-51-58}
\end{equation}
になる。\(\sin  \)の中については以下の変形をした。
\begin{equation}\begin{split}
\sin \l(\drac{n+m}{n}\drac{2\l(n-l+1\r)-1}{2}\pi \r)= {} & \sin \l[\l(n+m\r)\pi -\l(2l-1\r)\theta \r] \\
= {}                                                     & \sin \l(n+m\r)\pi \cos \l(2l-1\r)\theta -\cos \l(n+m\r)\pi \sin \l(2l-1\r)\theta \\
= {}                                                     & \l(-1\r)^{n+m+1}\sin \l(2l-1\r)\theta
\end{split}\label{eqn:2022-09-13_20-48-00}
\end{equation}
% である。
\begin{lemm}[]
\label{lem:2022-09-15_16-43-47}
\(n \in \mathbb{N} ,\theta  \in \mathbb{R} \)として以下の等式が成り立つ。
\begin{equation}
\sum_{l=1}^{n} \l(2l-1\r)\sin \l(2l-1\r)\theta =\drac{1}{2\sin ^2\theta }\l[-2n \cos 2n\theta \sin \theta +\sin 2n\theta \cos \theta \r]
\label{eqn:2022-09-13_20-52-53}
\end{equation}
\end{lemm}

% \lemref{lem:2022-09-15_16-43-47}も\secref{sec:補足;三角関数の総和公式について}で証明する。
\begin{prf}[\lemref{lem:2022-09-15_16-43-47}]
一般の\(\theta  \)に対して
\begin{align}
{}   & \sum_{l=1}^{n} \l(2l-1\r)\sin \l(2l-1\r)\theta =\drac{1}{\sin \theta }\sum_{l=1}^{n} \l(2l-1\r)\sin \l(2l-1\r)\theta \sin \theta \\
= {} & \drac{1}{\sin \theta }\sum_{l=1}^{n} \l(2l-1\r)\cdot \l(-\drac{1}{2}\r)\l(\cos \l(2l\theta \r)-\cos \l(2\l(l-1\r)\theta \r)\r) \\
= {} & -\drac{1}{2\sin \theta }\l[\sum_{l=1}^{n} \l\{\l(2l+1\r)\cos \l(2l\theta \r)-\l(2\l(l-1\r)+1\r)\cos \l(2\l(l-1\r)\theta \r)\r\}-\sum_{l=1}^{n}  2 \cos \l(2l\theta \r)\r] \\
= {} & -\drac{1}{2\sin \theta } \l[\l(2n+1\r)\cos 2n\theta -1-2\drac{\sin n\theta }{\sin \theta }\cos \l(n+1\r)\theta \r]\quad \text{(和の中抜けと\thmref{thm:2022-09-13_21-34-14}を用いた。)} \\
= {} & -\drac{1}{2\sin \theta }\l[2n \cos 2n\theta + \cos 2n\theta -1-\drac{2}{\sin \theta }\sin n\theta \cdot \l(\cos n\theta \cos \theta - \sin n\theta \sin \theta \r)\r] \\
= {} & -\drac{1}{2\sin \theta }\l[2n \cos 2n\theta +\cos 2n\theta -1-\drac{\cos \theta }{\sin \theta }\sin 2n\theta +2 \sin ^2 n\theta \r] \\
= {} & -\drac{1}{2\sin \theta }\l[2n \cos 2n\theta -\drac{\cos \theta}{\sin \theta }\sin 2n\theta \r]\qquad\text{(2倍角の公式より)} \\
= {} & \drac{1}{2\sin^2 \theta }\l[- 2n \cos 2n\theta \sin \theta +\sin 2n\theta \cos \theta \r]
\end{align}
である。
\end{prf}
これを用いると、\(\sin 2n\theta =\sin \l(n+m\r)\pi =0,\cos 2n\theta =\cos \l(n+m\r)\pi =\l(-1\r)^{n+m} \)より\eqref{eqn:2022-09-13_18-51-58}の式は
\begin{equation}\begin{split}
{}   & \drac{\l(-1\r)^{n+m+1}\pi }{2n}\cdot \drac{\l(-2n\r)\cdot \l(-1\r)^{n+m}\sin \theta }{2 \sin ^2\theta }= \drac{\pi }{2}\drac{1}{\sin \theta }=\drac{\pi }{2}\drac{1}{\sin \l(\orfrac{m}{n}\cdot \orfrac{\pi }{2}+\orfrac{\pi }{2 }\r)} = \drac{\pi }{2}\drac{1}{\cos \l(\orfrac{m}{n}\cdot \orfrac{\pi }{2}\r)}
\end{split}\label{eqn:2022-09-13_21-01-18}
\end{equation}
と変形できる。よって示された。
\end{prf}
\begin{thm}[0から\(\orfrac{\pi }4 \)までの定積分]
\label{thm:2022-09-13_15-56-40}
\(m \in \mathbb{Z},n \in \mathbb{N} \)として、\(\l|m\r|<n \)かつ\(m>0 \)のもとで以下の等式が成り立つ。
\begin{equation}
\int_{0}^{\drac{\pi }{4}}\tan ^{\drac{m}{n}}\theta  \dd{\theta }=\drac{\pi }{4 \cos \l(\orfrac{m}{n}\cdot \orfrac{\pi }{2}\r)}-\drac{1}{2}\sum_{k=1}^{n} \cos \l(n+m\r)\theta _{n,k}\log \l(2\l(1-\cos \theta _{n,k}\r)\r)
\label{eqn:2022-09-13_21-05-20}
\end{equation}

\end{thm}
\begin{prf}[]
\begin{equation}
f\l(\drac{\pi }{4}\r)=\begin{multlined}[t]
                      \sum_{k=1}^{n} \l[-\drac{1}{2}\cos \l(n+m\r)\theta _{n,k}\log\l(2\l(1-\cos \theta _{n,k}\r)\r) + \sin \l(n+m\r)\theta _{n,k}\arctan \l(\drac{1 - \cos \theta _{n,k}}{\sin \theta _{n,k}}\r)\r]
                      \end{multlined}
\label{eqn:2022-09-13_21-10-25}
\end{equation}
である。ここで
\begin{equation}
\drac{1-\cos \theta _{n,k}}{\sin \theta _{n,k}}=\drac{2 \sin ^2 \orfrac{\theta _{n,k}}{2}}{2 \sin \orfrac{\theta _{n,k}}{2}\cos \orfrac{\theta _{n,k}}{2}}=\tan \drac{\theta _{n,k}}{2}
\label{eqn:2022-09-13_21-12-33}
\end{equation}
であり、\(1\leq k\leq n \)より\(0<\theta _{n,k}<\pi  \)であるから\(0<\drac{\theta _{n,k}}{2}<\drac{\pi }{2} \)。よって以下が成り立つ。
\begin{equation}\begin{split}
{}   & \int_{0}^{\drac{\pi }{4}} \tan ^{\drac{m}{n}}\theta \dd{\theta }=f\l(\drac{\pi }{4}\r)-f\l(0\r) \\
= {} & \sum_{k=1}^{n} \l\{-\drac{1}{2}\cos \l(n+m\r)\theta _{n,k}\log \l[2\l(1-\cos \theta _{n,k}\r)\r]+\drac{1}{2}\l(\pi -\theta _{n,k}\r)\sin \l(n+m\r)\theta _{n,k}\r\} \\
= {} & \drac{\pi }{4 \cos \l(\orfrac{m}{n}\cdot \orfrac{\pi }{2}\r)}-\drac{1}{2}\sum_{k=1}^{n} \cos \l(n+m\r)\theta _{n,k}\log \l[2\l(1-\cos \theta _{n,k}\r)\r]
\end{split}\label{eqn:2022-09-13_21-16-00}
\end{equation}
最後の変形で第2項の値は\thmref{thm:2022-09-13_15-53-45}の証明の結果の半分になることを用いて求めた。

\end{prf}

% \section{補足;三角関数の総和公式について}
% \label{sec:補足;三角関数の総和公式について}







% \section{\(m<0 \)のとき}
% \label{sec:2022-09-15_18-07-36}
\subsection{指数の分子が負のとき}
\label{ssc:指数の分子が負のとき}

以下、簡単のため広義積分を\(\lim\)を使わずに表す。また、ややこしさを避けるために\(m<0 \)とするのではなく「\(m>0 \)として\(\tan ^{\drac{-m}{n}}\theta  \)の積分を求める」ことによって負数に拡張していることに注意。
\begin{thm}[]
\label{thm:2022-09-15_19-22-53}
指数の分子が負の場合も\thmref{thm:2022-09-13_15-53-45}が成立する。
\end{thm}
\begin{prf}[]
\(m>0 \)として指数を\(\drac{-m}{n} \)とする。
\begin{equation}\begin{split}
  \int_{0}^{\drac{\pi }{2}} \tan ^{\drac{-m }{n}}\theta \dd{\theta }
 = & {} \int_{\drac{\pi }{2}}^{0} \tan ^{\drac{-m}{n}}\l(\drac{\pi }{2}-\varphi \r)\dd{\varphi }\quad (\varphi =\drac{\pi }{2}-\theta と置換)\\
 = {}& \int_{0}^{\drac{\pi }{2}} \tan ^{\drac{m}{n}}\varphi \dd{\varphi }=\drac{\pi }{4\cos \l(\orfrac{m}{n}\cdot \orfrac{\pi }{2}\r)}=\drac{\pi }{4 \cos \l(\orfrac{-m}{n}\cdot \orfrac{\pi }{2}\r)}
\end{split}\label{eqn:2022-09-15_19-23-39}
\end{equation}
\end{prf}

\begin{thm}[]
指数の分子が負の場合も\thmref{thm:2022-09-13_15-56-40}が成立する。
\end{thm}

\begin{prf}[]
\(m>0 \)として指数を\(\drac{-m}{n} \)とする。
\begin{equation}\begin{split}
 \int_{0}^{\drac{\pi }{4}} \tan ^{\drac{-m}{n}}\theta \dd{\theta }
 = {}& \int_{\drac{\pi }{2}}^{\drac{\pi }{4}} \tan ^{\drac{-m}{n}}\varphi \dd{\varphi }\quad (\varphi =\drac{\pi }{2}-\theta と置換)\\
 = {}& \int_{\drac{\pi }{4}}^{\drac{\pi }{2}} \tan ^{\drac{m}{n}}\varphi  \dd{\varphi }=\int_{0}^{\drac{\pi }{2}} \tan ^{\drac{m}{n}}\varphi  \dd{\varphi }-\int_{0}^{\drac{\pi }{4}} \tan ^{\drac{m}{n}}\varphi  \dd{\varphi }\\
 = {}& \drac{\pi }{2 \cos \l(\orfrac{m}{n}\cdot \orfrac{\pi }{2}\r)}-\l[\drac{\pi }{4 \cos \l(\orfrac{m}{n}\cdot \orfrac{\pi }{2}\r)}-\drac{1}{2}\sum_{k=1}^{n} \cos \l(n+m\r)\theta _{n,k}\log \l(2\l(1- \cos \theta _{n,k}\r)\r)\r]
\end{split}\label{eqn:2022-09-15_18-19-05}
\end{equation}
である。ここで三角関数の和積公式より
\begin{equation}\begin{split}
 {}& \cos \l(n-m\r)\theta _{n,k}+ \cos \l(n+m\r)\theta _{n,k}=2 \cos n\theta _{n,k}
\cos n\theta _{n,k}\end{split}\label{eqn:2022-09-15_18-26-53}
\end{equation}
であり\(2k-1 \)は奇数なので
\begin{equation}
\cos  n\theta _{n,k}=\cos n\cdot \drac{2k-1}{2n}\pi =\cos \drac{2k-1}{2}\pi =0
\label{eqn:2022-09-15_18-28-39}
\end{equation}
よって
\begin{equation}
\cos \l(n+m\r)\theta _{n,k}=- \cos \l(n-m\r)\theta _{n,k}
\label{eqn:2022-09-15_18-30-05}
\end{equation}
以上より
\begin{equation}\begin{split}
 \int_{0}^{\drac{\pi }{4}} \tan ^{\drac{-m}{n}}\theta \dd{\theta }={} &  \drac{\pi }{4 \cos \l(\orfrac{m}{n}\cdot \orfrac{\pi }{2}\r)}+\drac{1}{2}\sum_{k=1}^{n} \l( -\cos \l(n-m\r)\theta _{n,k}\r)\cdot \log \l(2\l(1- \cos \theta _{n,k}\r)\r)\\
 ={}&\drac{\pi }{4 \cos \l(\orfrac{-m}{n}\cdot \orfrac{\pi }{2}\r)}-\drac{1}{2}\sum_{k=1}^{n} \cos \l(n-m\r)\theta _{n,k}\cdot \log \l(2\l(1- \cos \theta _{n,k}\r)\r)
\end{split}\label{eqn:2022-09-15_18-30-49}
\end{equation}
であり、指数の分子が負の場合も\thmref{thm:2022-09-13_15-56-40}を適用可能なことが示された。
\end{prf}
以上の議論により、\(\l|m\r|<n \)すなわち絶対値が\(1 \)より小さい有理数\(r \)に対して\(\tan ^r \theta  \)の積分を求められるという事実とその方法が得られた。


\section{\(\l|m\r|> n \)のとき}
\label{sec:その他のケース}
\subsection{指数の分子が正のとき}
\label{ssc:2022-09-15_20-20-36}

ここまでの議論では\(\l|m\r|<n \)のもののみを考えていた。ここからは\(\l|m\r|>n \)すなわち指数の絶対値が\(1 \)より大きくなるものを取り扱う。とはいえ\secref{sec:不定積分の証明}の部分分数分解の議論が成り立たないのでそのままの形で\(\l|m\r|<n \)の公式を用いることは不可能である(desmosでも実験してみたが違う値になってしまうようである)。そこで「次数を下げる」といういわゆる「\(\tan^n  \theta \)の積分」を解く方法を参考にしてみる。\(\l(\tan \theta \r)'=1+\tan ^2 \theta  \)の公式をもとにしている。
\begin{myburgundybox}{\(\tan ^n \theta \)の積分}
\(n \in \mathbb{N}\)とする。整式の割り算によって
\begin{equation}
\tan ^n \theta =\l(1+\tan ^2 \theta \r)p\l(\tan \theta \r)+a \tan \theta +b
\label{eqn:2022-09-15_20-03-20}
\end{equation}
(\(p \l(x\r)\)を\(x \)の整式とした。)と変形することによって\(P\l(x\r)=\int p\l(x\r) \dd{x}\)として
\begin{equation}
\int_{}^{} \tan ^n \theta \dd{\theta }=P\l(\tan \theta \r)-a \log \l|\cos \theta \r|+b\theta +C\quad (Cは積分定数)
\label{eqn:2022-09-15_20-07-58}
\end{equation}
として積分することが可能。
	\end{myburgundybox}
% \(\drac{m}{n} \)を整式の割り算をすると積分することが可能である。
試行錯誤の結果、以下のようにすると積分可能であることがわかった。
\begin{myburgundybox}{\(\tan ^{\drac{m}{n}}\theta  \)の積分}
	\(m>0,\l|m\r|>n ,\drac{m}{n} =2p+q\)(\(p \in \mathbb{N},-1<q<1, qは有理数 \))とする。
	\begin{equation}
	x^p=\l(1+x\r)p\l(x\r)+b
	\label{eqn:2022-09-15_20-28-36}
	\end{equation}
	(\(p\l(x\r) \)を\(x \)の整式とした)として
\begin{align}
\tan ^{\drac{m}{n}}\theta=& {} \tan ^{2p+q}\theta =\l(\tan ^2\theta \r)^p\tan ^{q}\theta \\
= {}& \l\{\l(1+\tan ^2 \theta \r)p\l(\tan ^2 \theta \r)+b\r\}\tan ^q \theta \\
= {}& \l(1+\tan ^2 \theta \r)\l[p\l(\tan ^2 \theta \r)\tan ^q \theta \r]+b \tan ^q \theta 
\label{eqn:2022-09-15_20-22-55}
\end{align}
と変形する。すると\(P\l(x\r)=\int_{}^{} p\l(x^2\r)x^q\dd{x} \)として
\begin{equation}
\int_{}^{} \tan ^{\drac{m}{n}}\theta \dd{\theta }=P\l(\tan \theta \r)+b \int_{}^{} \tan ^q \theta \dd{\theta }
\label{eqn:2022-09-15_20-38-09}
\end{equation}
として積分することが可能である。なお計算によって\(P,b \)の具体的な形を求めることができるので
\begin{equation}\begin{split}
\int_{}^{} \tan ^{\drac{m}{n}}\theta \dd{\theta }= {}& \sum_{k=0}^{p-1} \l(-1\r)^{k+p-1}\drac{\tan ^{2k+q+1}\theta }{2k+q+1}+\l(-1\r)^p \int_{}^{} \tan ^q \theta \dd{\theta }
\end{split}\label{eqn:2022-09-16_12-45-57}
\end{equation}
とすることも可能である。
\end{myburgundybox}
新しい受験数学の定石がまたひとつできた。
\subsection{指数の分子が負のとき}
\label{ssc:2022-09-15_20-48-30}
ここからは煩雑な分数表記を避けるために\(\cot \theta =\drac{1}{\tan \theta } \)を用いて表す。
\begin{myburgundybox}{\(\cot \)の微分公式}
\begin{equation}
\l(\cot \theta \r)'=-\l(1+\cot ^2 \theta \r)
\label{eqn:2022-09-15_20-54-32}
\end{equation}
\end{myburgundybox}
\begin{prf}[]
\begin{equation}\begin{split}
\l(\cot \theta \r)'= {}& \l(\drac{\cos \theta }{\sin \theta }\r)'=\drac{-\cos ^2 \theta- \sin ^2 \theta  }{\sin ^2 \theta }=-\l(1+\cot^2 \theta \r)
\end{split}\label{eqn:2022-09-15_20-55-16}
\end{equation}
\end{prf}
\(\cot \)を用いて\(\tan ^{\drac{-m}{n}}\theta  \)の積分を求めてみよう。引き続き\(m>0 \)のまま\(\tan ^{\drac{-m}{n}}\theta  \)の積分を求める方法をとる。
\begin{myburgundybox}{\(\tan ^{\drac{-m}{n}}\theta  \)の積分}
	\(m>0,\l|m\r|>n ,\drac{m}{n} =2p+q\)(\(p \in \mathbb{N},-1<q<1, qは有理数 \))とする。
	\begin{equation}
	x^p=\l(1+x\r)p\l(x\r)+b
	\end{equation}
	(\(p\l(x\r) \)を\(x \)の整式とした)として
\begin{align}
\tan ^{\drac{-m}{n}}\theta=\cot ^{\drac{m}{n}}\theta =& {} \cot ^{2p+q}\theta =\l(\cot ^2\theta \r)^p\cot ^{q}\theta \\
= {}& \l\{\l(1+\cot ^2 \theta \r)p\l(\cot ^2 \theta \r)+b\r\}\cot ^q \theta \\
= {}& \l(1+\cot ^2 \theta \r)\l[p\l(\cot ^2 \theta \r)\cot ^q \theta \r]+b \cot ^q \theta 
\end{align}
と変形する。すると\(P\l(x\r)=\int_{}^{} p\l(x^2\r)x^q\dd{x} \)として
\begin{equation}
\int_{}^{} \tan ^{\drac{m}{n}}\theta \dd{\theta }=-P\l(\tan \theta \r)+b \int_{}^{} \tan ^{-q} \theta \dd{\theta }
\end{equation}
として積分することが可能である。なお計算によって\(P,b \)の具体的な形を求めることができるので
\begin{equation}\begin{split}
	\int_{}^{} \tan ^{\drac{-m}{n}}\theta \dd{\theta }= {}& \sum_{k=0}^{p-1} \l(-1\r)^{k+p}\drac{\tan ^{2k+q+1}\theta }{2k+q+1}+\l(-1\r)^p \int_{}^{} \tan ^q \theta \dd{\theta }
	\end{split}
	\end{equation}
	とすることも可能である。
\end{myburgundybox}
% \subsection{}

以上により任意の有理数\(r \)に対して\(\tan ^{r} \theta \)を初等関数の範囲で具体的に積分する方法が得られた。
\section{今週の積分#\!100の問題}
\label{sec:元ネタ}

\begin{myburgundybox}{今週の積分#\!100の問題}
\begin{equation}
\int_{0}^{\drac{\pi }{4}} \sqrt{\tan x } \dd{x }
\label{eqn:2022-09-15_06-58-09}
\end{equation}
\end{myburgundybox}

\begin{mygraybox}{解}
\begin{equation}\begin{split}
{}   & \int_{0}^{\drac{\pi }{4}} \sqrt{\tan x }\dd{x }=\int_{0}^{\drac{\pi }{4}} \tan ^{\drac{1}{2}}x \dd{x } \\
= {} & \drac{\pi }{4 \cos \l(\orfrac{1}{2}\cdot \orfrac{\pi }{2}\r)}-\drac{1}{2}\sum_{k=1}^{2} \cos\l[ \l(2+1\r)\cdot \drac{2k-1}{2\cdot 2}\pi \r]\log \l(2\l(1- \cos \drac{2k-1}{2\cdot 2}\pi \r)\r) \\
= {} & \drac{\pi }{2\sqrt{2}}-\drac{1}{2}\l(\cos \drac{3}{4}\pi \log \l(2-\sqrt{2}\r)+\cos \drac{9}{4}\pi \log \l(2+\sqrt{2}\r) \r) \\
= {} & \drac{\sqrt{2}}{4}\pi -\drac{\sqrt{2}}{4}\log \l(1+\sqrt{2}\r)^2 \\
= {} & \drac{\sqrt{2}}{4}\pi -\drac{\sqrt{2}}{2}\log \l(1+\sqrt{2}\r)……(答)
\end{split}\label{eqn:2022-09-15_06-59-59}
\end{equation}

\end{mygraybox}
\begin{myburgundybox}{例題}
次の積分を求めよ。
\begin{equation}
\int_{0}^{\drac{\pi }{4}}\tan ^{\drac{3}{2}}\theta  \dd{\theta }
\label{eqn:2022-09-13_23-02-24}
\end{equation}
\end{myburgundybox}

\begin{mygraybox}{解法;偶数に寄せて次数を減らす!}
\begin{equation}\begin{split}
{}   & \int_{0}^{\drac{\pi }{4}} \tan ^{\drac{3}{2}}\theta \dd{\theta } \\
= {} & \int_{0}^{\drac{\pi }{4}} \l(1+\tan ^2\theta \r)\tan ^{-\drac{1}{2}}\theta -\tan ^{-\drac{1}{2}}\theta \dd{\theta } \\
= {} & \l[2 \tan ^{\drac{1}{2}}\theta \r]_{0}^{\drac{\pi }{4}}-\int_{0}^{\drac{\pi }{4}} \tan ^{-\drac{1}{2}}\theta \dd{\theta } \\
= {} & 2-\drac{\pi }{2\sqrt{2}}+\drac{1}{2}\sum_{k=1}^{2} \cos \l[\l(2-1\r)\drac{2k-1}{2\cdot 2}\pi \r]\log \l[2\l(1-\cos \drac{2k-1}{2\cdot 2}\r)\pi \r] \\
= {} & 2-\drac{\sqrt{2}}{4}\pi +\drac{1}{2}\l(\cos \drac{\pi }{4}\log \l[2\l(1-\drac{1}{\sqrt{2}}\r)\r]+\cos \drac{3}{4}\pi \log \l[2\l(1+\drac{1}{\sqrt{2}}\r)\r]\r) \\
= {} & 2-\drac{\sqrt{2}}{4}\pi +\drac{1}{2}\l\{\drac{1}{\sqrt{2}}\log \l(2-\sqrt{2}\r)-\drac{1}{\sqrt{2}}\log \l(2+\sqrt{2}\r)\r\} \\
= {} & 2-\drac{\sqrt{2}}{4}\pi -\drac{\sqrt{2}}{4}\log \l|\drac{\sqrt{2}+1}{\sqrt{2}-1}\r| \\
= {} & 2-\drac{\sqrt{2}}{4}\pi -\drac{\sqrt{2}}{2}\log \l(1+\sqrt{2}\r)……(答)
\end{split}\label{eqn:2022-09-13_23-03-59}
\end{equation}
\end{mygraybox}

\begin{myburgundybox}{例題}
次の積分を求めよ。
\begin{equation}
\int_{\drac{\pi }{4}}^{\drac{\pi }{2}} \tan ^{-\drac{5}{2}}\theta \dd{\theta }
\label{eqn:2022-09-15_07-15-26}
\end{equation}
\end{myburgundybox}


\begin{mygraybox}{解法;置換積分で範囲を変える}
\(\varphi =\drac{\pi }{2}-\theta  \)とする。\(\dd{\theta }=-\dd{\varphi } \)より
\begin{equation}\begin{split}
{}   & \int_{\drac{\pi }{4}}^{\drac{\pi }{2}} \tan ^{-\drac{5}{2}}\theta \dd{\theta }=\int_{\drac{\pi }{4}}^{0} \tan ^{-\drac{5}{2}}\l(\drac{\pi }{2}-\varphi\r) \dd{\varphi }=\int_{0}^{\drac{\pi }{4}} \tan ^{\drac{5}{2} }\varphi\dd{\varphi} \\
= {} & \int_{0}^{\drac{\pi }{4}} \l(1+\tan ^2\varphi \r)\cdot \tan ^{\drac{1}{2}}\varphi \dd{\varphi }-\int_{0}^{\drac{\pi }{4}} \tan ^{\drac{1}{2}}\varphi \dd{\varphi } \\
= {} & \l[ \drac{2}{3}\tan ^{\drac{3}{2}}\varphi  \r]_{0}^{\drac{\pi }{4 }}-\l(\drac{\sqrt{2}}{4}\pi -\drac{\sqrt{2}}{2}\log \l(1+\sqrt{2}\r)\r) \\
= {} & \drac{2}{3}-\drac{\sqrt{2}}{4}\pi +\drac{\sqrt{2}}{2}\log \l(1+\sqrt{2}\r)……(答)
\end{split}\label{eqn:2022-09-15_07-23-59}
\end{equation}
\end{mygraybox}












\section{応用例}
\label{sec:応用例}
\begin{myburgundybox}{公式}
\begin{equation}
\int_{0}^{1} \drac{1}{1+x^n}\dd{x}=\drac{2}{n}\int_{0}^{\drac{\pi }{4}}\tan ^{\drac{2}{n}-1}\theta  \dd{\theta }
\label{eqn:2022-09-15_07-54-15}
\end{equation}
\end{myburgundybox}
\begin{mygraybox}{解答}
\(x=\tan ^{\drac{2}{n}}\theta  \)とおいて置換積分する。
\end{mygraybox}
ここからさらに一般化して\(\int_{0}^1\drac{\dd{x}}{1+x^n}  \)の積分を\(n \)で表すこともできる。(\(n>1 \)とする。)
\begin{equation}
\int_{0}^{1} \drac{\dd{x}}{1+x^n}=\drac{1}{n}\l(\drac{\pi }{2 \sin \orfrac{\pi }{n}}-\sum_{k=1}^{n} \cos 2\theta _{n,k}\log \l[2\l(1- \cos \theta _{n,k}\r)\r]\r)
\label{eqn:2022-09-15_22-35-50}
\end{equation}
裏技公式的に暗記しておいても損はないかもしれない。だがあえて一般化しないメリットもある。以下の問題を見ていただきたい。
% 普通に解くと大変なことになるあの問題も(まあまあ)簡単に解ける。
\begin{myburgundybox}{問題}
次の積分を求めよ。
\begin{equation}
\int_{0}^{\drac{\pi }{4}} \drac{\dd{x}}{1+x^6}
\label{eqn:2022-09-15_08-10-10}
\end{equation}
\end{myburgundybox}
\begin{mygraybox}{解答}
公式を用いる。
\begin{equation}\begin{split}
{}   & \int_{0}^{\drac{\pi }{6}} \drac{\dd{x}}{1+x^6}=\drac{2}{6}\int_{0}^{\drac{\pi }{4}} \tan ^{\drac{2}{6}-1}\theta \dd{\theta }=\drac{1}{3}\int_0^{\drac{\pi }{4}}\tan ^{-\drac{2}{3}}\theta  \dd{\theta } \\
= {} & \drac{1}{3}\l\{\drac{\pi }{4 \cos \l(- \orfrac{2}{3}\cdot \orfrac{\pi }{2}\r)}-\drac{1}{2}\sum_{k=1}^{3} \cos \l[\l(-2+3\r)\drac{2k-1}{2\cdot 3}\pi \r]\log \l(2\l(1-\cos \drac{2k-1}{2\cdot 3}\r)\pi \r)\r\} \\
= {} & \drac{1}{3}\l\{\drac{\pi }{4\cdot \orfrac{1}{2}}-\drac{1}{2}\l[\drac{\sqrt{3}}{2}\log \l(2-\sqrt{3}\r)+0+\l(-\drac{\sqrt{3}}{2}\r)\log \l(2+\sqrt{3}\r)\r]\r\} \\
= {} & \drac{1}{3}\l\{\drac{\pi }{2}+\drac{\sqrt{3}}{4}\log \l(\drac{2+\sqrt{3}}{2-\sqrt{3}}\r)\r\} \\
= {} & \drac{\pi }{6}+\drac{\sqrt{3}}{6}\log \l(2+\sqrt{3}\r)……(答)
\end{split}\label{eqn:2022-09-15_08-11-01}
\end{equation}
\end{mygraybox}
実は途中で\(\drac{2}{6}-1 =-\drac{4}{6}\)を\(-\drac{2}{3} \)に約分することによって計算する項数が半分になっている。おわかりいただけただろうか。ちなみにこの問題、普通に解くと部分分数分解が難しい骨のある問題である。
\section{おわりに}
\label{sec:おわりに}
任意の有理数\(r \)に対して\(\tan ^r \theta  \)が初等関数の範囲で積分可能であることを示し、具体的な求め方を考えてみました。はじめは\secref{sec:不定積分の証明}の不定積分の内容のみで済ませる予定でしたが
\begin{itemize}
	\item 特定の範囲の定積分の値が綺麗になった
	\item 任意の有理数に対して示したくなった
	\item 定石として汎用的に使う方法を思いついてしまった
\end{itemize}
ことから非常に長い内容になってしまいました。欲を言うと実数全体について示したくなってしまったのですが、今は近似分数列が収束するということにして納得しています\footnote{この辺もいつかちゃんと証明したいと思っています。\label{fot:2022-09-16_12-00-22}}。

論旨から外れるので書きませんでしたが、もし実数全体について示すことができれば\(\Gamma  \)関数の相反公式等を示すことができます。\\

もしこの記事の内容が一般的に知られる世界になったら、大学受験の積分問題の定石の一つとして「\(\tan ^{\drac{m}{n}}\theta  \)の積分」なるものが掲載されることがあるかな、などと妄想しております。が、導出が大変なのでまあないでしょう\footnote{もしかしたらもっと簡単に求められる方法があるかもしれません…そうなったら載るかも…?\label{fot:2022-09-15_21-46-36}}。\\

長い内容でしたがここまでお読みいただきありがとうございました。
\section{おまけ}
\label{sec:おまけ}
\(\theta _{n,k}=\drac{2k-1}{2n}\pi  \)とする。
\begin{align}
& \int \drac{\dd{x}}{1+x^n} = \drac{2}{n}\sum_{k=1}^{n} \l[-\drac{1}{2}\cos 2\theta _{n,k}\log \l(x-2\sqrt{x}\cos \theta _{n,k}+1\r)+\sin 2\theta _{n,k}\arctan \l(\drac{\sqrt{x}-\cos \theta _{n,k}}{\sin \theta _{n,k}}\r)\r]+C
\label{eqn:2022-09-18_10-43-29}\\
& \int_{}^{} \sqrt{\tan \theta }\dd{\theta } = -\drac{\sqrt{2}}{4}\log \l(\drac{\tan \theta +\sqrt{2 \tan \theta }+1}{\tan \theta -\sqrt{2 \tan \theta }+1}\r)+\drac{\sqrt{2}}{2}\l(\arctan \l(\sqrt{2 \tan \theta }+1\r)+\arctan \l(\sqrt{2 \tan \theta }-1\r)\r)+C
\label{eqn:2022-09-18_10-45-43}\\
 {}& \int_{}^{} \drac{1}{\sqrt{\tan \theta }}\dd{\theta }=\drac{\sqrt{2}}{4}\log \l(\drac{\tan \theta +\sqrt{2 \tan \theta }+1}{\tan \theta -\sqrt{2 \tan \theta }+1}\r)+\drac{\sqrt{2}}{2}\l(\arctan \l(\sqrt{2 \tan \theta }+1\r)+\arctan \l(\sqrt{2 \tan \theta }-1\r)\r)+C
\label{eqn:2022-09-18_11-15-59}\\
 {}& \drac{1}{z^n+1}=-\drac{1}{n}\sum_{k=1}^{n} \drac{e\l(\orfrac{2k-1}{n}\pi \r)}{z-e\l(\orfrac{2k-1}{n}\pi \r)}
\label{eqn:2022-09-18_12-43-27}\\
 {}& \drac{1}{z^n-1}=\drac{1}{n}\sum_{k=1}^{n} \drac{e\l(\orfrac{2k\pi }{n}\r)}{z-e\l(\orfrac{2k\pi }{n}\r)}=\drac{1}{n}\sum_{k=1}^{n} \drac{\exp\l(i\orfrac{2k\pi }{n}\r)}{z-\exp\l(i\orfrac{2k\pi }{n}\r)}
\label{eqn:2022-09-18_12-45-02}
\end{align}
\section{参考文献}
\label{sec:参考文献}


YouTubeヨビノリたくみさんのチャンネル \href{https://www.google.com/url?sa=t&rct=j&q=&esrc=s&source=video&cd=&cad=rja&uact=8&ved=2ahUKEwiar7fIzpb6AhUDw4sBHU28DxMQFnoECAkQAg&url=https://www.youtube.com/channel/UCqmWJJolqAgjIdLqK3zD1QQ&usg=AOvVaw3clbmmW3q8O9-Fgp4nIJlw}{予備校のノリで学ぶ「大学の数学・物理」}\par
 

\parbox{\linewidth}{\url{https://www.google.com/url?sa=t&rct=j&q=&esrc=s&source=video&cd=&cad=rja&uact=8&ved=2ahUKEwiar7fIzpb6AhUDw4sBHU28DxMQFnoECAkQAg&url=https://www.youtube.com/channel/UCqmWJJolqAgjIdLqK3zD1QQ&usg=AOvVaw3clbmmW3q8O9-Fgp4nIJlw}}








% \begin{thm}[\(\Gamma  \)関数の相反公式]
% \label{thm:2022-09-13_19-27-35}

% \end{thm}
% \(a+b=c+d \)より\(\drac{1}{2}+\drac{1}{3}=\drac{1}{4} \)である。










\end{document}